\documentclass[a4paper, 11pt]{extarticle}
%%% Работа с русским языком
\usepackage{cmap}					% поиск в PDF
\usepackage{mathtext} 				% русские буквы в формулах
\usepackage[T2A]{fontenc}			% кодировка
\usepackage[utf8]{inputenc}			% кодировка исходного текста
\usepackage[english,russian]{babel}	% локализация и переносы
%\usepackage[tmargin=4cm,bmargin=4cm,lmargin=2cm,rmargin=2cm,asymmetric,headheight=50pt,bindingoffset=1cm]{geometry}
\usepackage[tmargin=1.5cm,bmargin=1.5cm,lmargin=2cm,rmargin=1.5cm,headheight=50pt]{geometry}
\usepackage{comment} % enables the use of multi-line comments (\ifx \fi) 
\usepackage{lipsum} %This package just generates Lorem Ipsum filler text. 
\usepackage{fullpage} % changes the margin
%\usepackage[a4paper, total={7in, 10in}]{geometry}
\usepackage[fleqn]{amsmath}
\usepackage{amssymb,amsthm}  % assumes amsmath package installed
\newtheorem{theorem}{Theorem}
\newtheorem{corollary}{Corollary}
\usepackage{graphicx}
\usepackage{tikz}
\usetikzlibrary{arrows}
\usepackage{verbatim}
%\usepackage[numbered]{mcode}
\usepackage{float}
\usepackage{tikz}
    \usetikzlibrary{shapes,arrows}
    \usetikzlibrary{arrows,calc,positioning}

    \tikzset{
        block/.style = {draw, rectangle,
            minimum height=1cm,
            minimum width=1.5cm},
        input/.style = {coordinate,node distance=1cm},
        output/.style = {coordinate,node distance=4cm},
        arrow/.style={draw, -latex,node distance=2cm},
        pinstyle/.style = {pin edge={latex-, black,node distance=2cm}},
        sum/.style = {draw, circle, node distance=1cm},
    }
\usepackage{xcolor}
\usepackage{mdframed}
\usepackage[shortlabels]{enumitem}
%\usepackage{indentfirst}
\usepackage{hyperref}

\hypersetup{
    colorlinks=true,
    linkcolor=blue,
    filecolor=magenta,      
    urlcolor=cyan,
}
\renewcommand{\thesubsection}{\thesection.\alph{subsection}}

\newenvironment{problem}[2][Problem]
    { \begin{mdframed}[backgroundcolor=gray!20] \textbf{#1 #2} \\}
    {  \end{mdframed}}

% Define solution environment
\newenvironment{solution}
    {\textit{Solution:}}
    {}

\renewcommand{\qed}{\quad\qedsymbol}


\usepackage{fancyhdr}
\usepackage{emptypage}
\fancyhead{}
\fancyfoot{}
\fancyhead[LE,RO]{\includegraphics [width=.3\textwidth] {logo.png}}
\fancyfoot[LE,RO]{\thepage}
\pagestyle{fancy}
\graphicspath{{../../tex_templates/figures/}}

% code listing settings
\usepackage{listings}
\lstset{
    language=Python,
    basicstyle=\ttfamily\small,
    aboveskip={1.0\baselineskip},
    belowskip={1.0\baselineskip},
    columns=fixed,
    extendedchars=true,
    breaklines=true,
    tabsize=4,
    prebreak=\raisebox{0ex}[0ex][0ex]{\ensuremath{\hookleftarrow}},
    frame=lines,
    showtabs=false,
    showspaces=false,
    showstringspaces=false,
    keywordstyle=\color[rgb]{0.627,0.126,0.941},
    commentstyle=\color[rgb]{0.133,0.545,0.133},
    stringstyle=\color[rgb]{01,0,0},
    numbers=left,
    numberstyle=\small,
    stepnumber=1,
    numbersep=10pt,
    captionpos=t,
    escapeinside={\%*}{*)}
}



%%%%%%%%%%%%%%%%%%%%%%%%%%%%%%%%%%%%%%%%%%%%%%%%%%%%%%%%%%%%%%%%%%%%%%%%%%%%%%%%%%%%%%%%%%%%%%%%%%%%%%%%%%%%%%%%%%%%%%%%%%%%%%%%%%%%%%%%
\begin{document}
%Header-Make sure you update this information!!!!
%\noindent
%%%%%%%%%%%%%%%%%%%%%%%%%%%%%%%%%%%%%%%%%%%%%%%%%%%%%%%%%%%%

\noindent \LARGE{\textbf{Course: WebDev -- Django Start.}} \hfill  \\ 
%\vspace{0.25em}
\textbf{Homework - 4}  \hfill  \\
%Email: veralevel@gethu.edu \hfill ID: 123456789 \\

\noindent Theme: Introduction To Django \hfill  \\
Level: beginner\\
Instructor: Mikhail Nakonechnyi \\
Due Date: $6^{th}$ March, 2020 \\
\noindent\rule{6.257in}{2.8pt}
%%%%%%%%%%%%%%%%%%%%%%%%%%%%%%%%%%%%%%%%%%%%%%%%%%%%%%%%%%%%

\begin{problem}{1. Создание сайта. Этап 1 - Подготовка}
На начальном этапе необходимо провести некую подготовительную работу. Такая работа предполагает сбор информации о целевой аудитории и целях ресурса. Иначе говоря, вы должны придумать \textbf{ИДЕЮ}. Наш первый сайт будет блогом. Основываясь на результатах сбора информации, вы должны сделать вывод о том, какую тематику для своего сайта выбрать: новости (спорт, мода, наука...), тематический блог (кулинария, автомобили, шитье и тд). И следующим моментом будет придумать название, которое подчеркивает тематику сайта. \\
\textit{\textbf{Примечание:}} дополнительные ресурсы для помощи:
\begin{itemize}
\item \href{https://blog.bigtime.ventures/kak-uznat-interesy-tselevoy-auditorii-metodiki-instrumenty-podkhody}{Как узнать интересы целевой аудитории}
\item \href{https://www.ucraft.ru/blog/kak-sobrat-informatsiyu-o-tsa-dlya-sajta}{Как собрать информацию об аудитории}
\href{https://www.marketing.spb.ru/lib-research/methods/collect_and_analysis.htm}{Методы сбора информации и инструменты анализа}
\item \href{https://texterra.ru/blog/o-chem-vesti-blog-26-idey-dlya-nachinayushchikh.html}{О чем вести блог}
\item \href{https://ifish2.ru/temy-dlya-bloga/}{Темы для блога}
\item \href{https://texterra.ru/blog/7-tem-dlya-bloga-kotorye-zatsepyat-lyubogo.html}{Еще темы}
\item \href{https://livesurf.ru/zhurnal/5719-idei-dlya-vedeniya-bloga.html}{И еще темы}
\end{itemize}
\end{problem}
\noindent\rule{6.257in}{2.8pt}


%%%%%%%%%%%%%%%%%%%%%%%%%%%%%%%%%%%%%%%%%%%%%%%%%%%%%%%%%%%%

\begin{problem}{2. Создание сайта. Этап 2 - Контент}
Основываясь на своей тематике сайта, написать/сочинить как минимум 1 статью. Название статьи должно быть не более 200 символов. Статья должна содержать не менее 3000 символов (с пробелами). Для статьи можно найти несколько тематических картинков широкого формата. Для работы воспользуйтесь \textbf{Google Docs}. Для того, чтобы вести учет своих работ - создайте папку на гугл диске с названием сайта, а в ней папку \textbf{Статьи}. Самый простой вариант хранить статьи подставляя номер статьи перед ее названием. \\
\textbf{Контент} -- основа вашего сайта. Будь это авторские статьи или пользовательские статьи -- пользователь в первую очередь приходит за информацией, и мы должны в приятной форме её преподнести.
На этом этапе необходимо иметь представление, какие разделы будут на вашем сайте. Например если будет страница <<О сайте>> -- что на ней найдет посетитель?
Рекомендую проверять на орфографические ошибки (хотя бы при помощи MS Word). Остальные подробности о написании статей узнаете на занятии.\\
\textit{\textbf{Примечание:}} ссылку на гугл документ отправить мне в телеграме. Я не приму работу скопированную под чистую из интернета.
\end{problem}
\noindent\rule{6.257in}{2.8pt}
%%%%%%%%%%%%%%%%%%%%%%%%%%%%%%%%%%%%%%%%%%%%%%%%%%%%%%%%%%%%

\begin{problem}{3. Знакомство с Django}

\end{problem}
\noindent\rule{6.257in}{2.8pt}
\end{document}
 