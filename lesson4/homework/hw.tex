\documentclass[a4paper, 11pt]{extarticle}
%%% Работа с русским языком
\usepackage{cmap}					% поиск в PDF
\usepackage{mathtext} 				% русские буквы в формулах
\usepackage[T2A]{fontenc}			% кодировка
\usepackage[utf8]{inputenc}			% кодировка исходного текста
\usepackage[english,russian]{babel}	% локализация и переносы
%\usepackage[tmargin=4cm,bmargin=4cm,lmargin=2cm,rmargin=2cm,asymmetric,headheight=50pt,bindingoffset=1cm]{geometry}
\usepackage[tmargin=1.5cm,bmargin=1.5cm,lmargin=2cm,rmargin=1.5cm,headheight=50pt]{geometry}
\usepackage{comment} % enables the use of multi-line comments (\ifx \fi) 
\usepackage{lipsum} %This package just generates Lorem Ipsum filler text. 
\usepackage{fullpage} % changes the margin
%\usepackage[a4paper, total={7in, 10in}]{geometry}
\usepackage[fleqn]{amsmath}
\usepackage{amssymb,amsthm}  % assumes amsmath package installed
\newtheorem{theorem}{Theorem}
\newtheorem{corollary}{Corollary}
\usepackage{graphicx}
\usepackage{tikz}
\usetikzlibrary{arrows}
\usepackage{verbatim}
%\usepackage[numbered]{mcode}
\usepackage{float}
\usepackage{tikz}
    \usetikzlibrary{shapes,arrows}
    \usetikzlibrary{arrows,calc,positioning}

    \tikzset{
        block/.style = {draw, rectangle,
            minimum height=1cm,
            minimum width=1.5cm},
        input/.style = {coordinate,node distance=1cm},
        output/.style = {coordinate,node distance=4cm},
        arrow/.style={draw, -latex,node distance=2cm},
        pinstyle/.style = {pin edge={latex-, black,node distance=2cm}},
        sum/.style = {draw, circle, node distance=1cm},
    }
\usepackage{xcolor}
\usepackage{mdframed}
\usepackage[shortlabels]{enumitem}
%\usepackage{indentfirst}
\usepackage{hyperref}

\hypersetup{
    colorlinks=true,
    linkcolor=blue,
    filecolor=magenta,      
    urlcolor=cyan,
}
\renewcommand{\thesubsection}{\thesection.\alph{subsection}}

\newenvironment{problem}[2][Problem]
    { \begin{mdframed}[backgroundcolor=gray!20] \textbf{#1 #2} \\}
    {  \end{mdframed}}

% Define solution environment
\newenvironment{solution}
    {\textit{Solution:}}
    {}

\renewcommand{\qed}{\quad\qedsymbol}


\usepackage{fancyhdr}
\usepackage{emptypage}
\fancyhead{}
\fancyfoot{}
\fancyhead[LE,RO]{\includegraphics [width=.3\textwidth] {logo.png}}
\fancyfoot[LE,RO]{\thepage}
\pagestyle{fancy}
\graphicspath{{../../tex_templates/figures/}}

% code listing settings
\usepackage{listings}
\lstset{
    language=Python,
    basicstyle=\ttfamily\small,
    aboveskip={1.0\baselineskip},
    belowskip={1.0\baselineskip},
    columns=fixed,
    extendedchars=true,
    breaklines=true,
    tabsize=4,
    prebreak=\raisebox{0ex}[0ex][0ex]{\ensuremath{\hookleftarrow}},
    frame=lines,
    showtabs=false,
    showspaces=false,
    showstringspaces=false,
    keywordstyle=\color[rgb]{0.627,0.126,0.941},
    commentstyle=\color[rgb]{0.133,0.545,0.133},
    stringstyle=\color[rgb]{01,0,0},
    numbers=left,
    numberstyle=\small,
    stepnumber=1,
    numbersep=10pt,
    captionpos=t,
    escapeinside={\%*}{*)}
}



%%%%%%%%%%%%%%%%%%%%%%%%%%%%%%%%%%%%%%%%%%%%%%%%%%%%%%%%%%%%%%%%%%%%%%%%%%%%%%%%%%%%%%%%%%%%%%%%%%%%%%%%%%%%%%%%%%%%%%%%%%%%%%%%%%%%%%%%
\begin{document}
%Header-Make sure you update this information!!!!
%\noindent
%%%%%%%%%%%%%%%%%%%%%%%%%%%%%%%%%%%%%%%%%%%%%%%%%%%%%%%%%%%%

\noindent \LARGE{\textbf{Course: WebDev -- Django Start.}} \hfill  \\ 
%\vspace{0.25em}
\textbf{Homework - 4}  \hfill  \\
%Email: veralevel@gethu.edu \hfill ID: 123456789 \\

\noindent Theme: Introduction To Django \hfill  \\
Level: beginner\\
Instructor: Mikhail Nakonechnyi \\
Due Date: $6^{th}$ March, 2020 \\
\noindent\rule{6.257in}{2.8pt}
%%%%%%%%%%%%%%%%%%%%%%%%%%%%%%%%%%%%%%%%%%%%%%%%%%%%%%%%%%%%

\begin{problem}{1. Создание сайта. Этап 1 - Подготовка}
На начальном этапе необходимо провести некую подготовительную работу. Такая работа предполагает сбор информации о целевой аудитории и целях ресурса. Иначе говоря, вы должны придумать \textbf{ИДЕЮ}. Наш первый сайт будет блогом. Основываясь на результатах сбора информации, вы должны сделать вывод о том, какую тематику для своего сайта выбрать: новости (спорт, мода, наука...), тематический блог (кулинария, автомобили, шитье и тд). И следующим моментом будет придумать название, которое подчеркивает тематику сайта. \\
\textit{\textbf{Примечание:}} дополнительные ресурсы для помощи:
\begin{itemize}
\item \href{https://blog.bigtime.ventures/kak-uznat-interesy-tselevoy-auditorii-metodiki-instrumenty-podkhody}{Как узнать интересы целевой аудитории}
\item \href{https://www.ucraft.ru/blog/kak-sobrat-informatsiyu-o-tsa-dlya-sajta}{Как собрать информацию об аудитории}
\href{https://www.marketing.spb.ru/lib-research/methods/collect_and_analysis.htm}{Методы сбора информации и инструменты анализа}
\item \href{https://texterra.ru/blog/o-chem-vesti-blog-26-idey-dlya-nachinayushchikh.html}{О чем вести блог}
\item \href{https://ifish2.ru/temy-dlya-bloga/}{Темы для блога}
\item \href{https://texterra.ru/blog/7-tem-dlya-bloga-kotorye-zatsepyat-lyubogo.html}{Еще темы}
\item \href{https://livesurf.ru/zhurnal/5719-idei-dlya-vedeniya-bloga.html}{И еще темы}
\end{itemize}
\end{problem}
\noindent\rule{6.257in}{2.8pt}


%%%%%%%%%%%%%%%%%%%%%%%%%%%%%%%%%%%%%%%%%%%%%%%%%%%%%%%%%%%%

\begin{problem}{2. Создание сайта. Этап 2 - Контент}
Основываясь на своей тематике сайта, написать/сочинить как минимум 1 статью. Название статьи должно быть не более 200 символов. Статья должна содержать не менее 3000 символов (с пробелами). Для статьи можно найти несколько тематических картинков широкого формата. Для работы воспользуйтесь \textbf{Google Docs}. Для того, чтобы вести учет своих работ - создайте папку на гугл диске с названием сайта, а в ней папку \textbf{Статьи}. Самый простой вариант хранить статьи подставляя номер статьи перед ее названием. \\
\textbf{Контент} -- основа вашего сайта. Будь это авторские статьи или пользовательские статьи -- пользователь в первую очередь приходит за информацией, и мы должны в приятной форме её преподнести.
На этом этапе необходимо иметь представление, какие разделы будут на вашем сайте. Например если будет страница <<О сайте>> -- что на ней найдет посетитель?
Рекомендую проверять на орфографические ошибки (хотя бы при помощи MS Word). Остальные подробности о написании статей узнаете на занятии.\\
\textit{\textbf{Примечание:}} ссылку на гугл документ отправить мне в телеграме. Я не приму работу скопированную под чистую из интернета.
\end{problem}
\noindent\rule{6.257in}{2.8pt}
%%%%%%%%%%%%%%%%%%%%%%%%%%%%%%%%%%%%%%%%%%%%%%%%%%%%%%%%%%%%
\begin{problem}{4. Команды Bash}
Некоторые команды для работы с командной строкой терминала в UNIX/Linux или терминалом powershell(Windows), которые необходимо знать и использовать все время для работы:
\begin{itemize}
\item \textbf{ls} - отображение файлов и папок находящихся в директории
\item \textbf{ls -l} - отображение файлов и папок с дополнительными параметрами
\item \textbf{ls -la} - отображение файлов и папок с дополнительными параметрами и скрытыми файлами
\item \textbf{cd PATH} - перейти в каталог по заданному пути 
\item \textbf{cd ..} - перейти в "родительский" каталог
\item \textbf{cd $\sim$} $\, $ - перейти в "домашнюю" директорию
\item \textbf{cd -} - вернуться на одно действие назад (местонахождение)
\item \textbf{cd path/to/documents/} - пример "пути"
\item \textbf{clear} - очистить окно терминала
\item \textbf{mkdir $FOLDER\_NAME$} - создать папку с названием
\item \textbf{mv path/to/file.txt newfile.html} - переместить файл / папку или переименовать файл / папку
\item \textbf{cp path/to/file.txt newfile.html} - скопировать файл или папку 
\item \textbf{rm path/to/file.txt} - удалить файл по адресу
\item \textbf{rm -rf directory/} - удалить папку
\item \textbf{touch index.html} - создать файл index.html
\end{itemize}
\end{problem}
\noindent\rule{6.257in}{2.8pt}

\begin{problem}{4. Установка Python}
\textit{\textbf{Примечание:}} инструкции и описания ниже относятся к терминалу Linux (в частности, к нашему IDE для работы).\\
Django написан на Python. Нам нужен Python, чтобы сделать что-нибудь в Django. Начнем с его установки! Мы хотим самую свежую версию Python 3. Для установки нужно ввести команду:
\begin{lstlisting}[language=Bash]
$ apt-get install python3
\end{lstlisting}
Если уже установлена версия 3.4 или более высокая, она должна подойти. Для проверки можно воспользоваться следующей командой:
\begin{lstlisting}[language=Bash]
$ python3 --version
\end{lstlisting}
\end{problem}
\noindent\rule{6.257in}{2.8pt}


\begin{problem}{5. Настройка виртуального окружения}
Перед установкой Django мы должны установить крайне полезный инструмент, который поможет содержать среду разработки в чистоте. Можно пропустить этот шаг, но лучше этого не делать. Использование лучших рекомендаций с самого начала убережёт от многих проблем в будущем!
Итак, создадим виртуальное окружение (оно также называется virtualenv). \textbf{Virtualenv} будет изолировать настройки Python/Django для каждого отдельного проекта. Это значит, что изменения одного сайта не затронут другие сайты, которые будут разрабатываться. 
Всё, что нужно сделать -- найти директорию, в которой мы создадим virtualenv. 
Мы будем использовать директорию нашего проекта в домашнем каталоге:
\begin{lstlisting}[language=Bash]
~/$ cd django
\end{lstlisting}
Мы создадим виртуальное окружение под именем \textbf{myenv}. В общем случае команда будет выглядеть так:
\begin{lstlisting}[language=Bash]
~/django/$ python3 -m venv myenv
\end{lstlisting}
Указанная выше команда создаст директорию $myenv$ (или другую, в зависимости от выбранного имени), которая будет содержать виртуальное окружение (по сути -- набор файлов и папок).
Для запуска виртуального окружения, необходимо выполнить следующую команду в папке $django$:
\begin{lstlisting}[language=Bash]
~/django/$ source myenv/bin/activate
\end{lstlisting}
Будет видно, что virtualenv запущено, когда появится префикс $(myenv)$ в начале командной строки. 
\begin{lstlisting}[language=Bash]
(myenv) ~/django/$ 
\end{lstlisting}
При работе с виртуальным окружением команда python будет автоматически обращаться к правильной версии языка, так что можно использовать просто python вместо python3.
Теперь мы будем хранить все важные зависимости в одном месте. 
\end{problem}
\noindent\rule{6.257in}{2.8pt}



\begin{problem}{5. Знакомство с Django}
Перейдем к установке Django!
\end{problem}
\noindent\rule{6.257in}{2.8pt}
\end{document}
 