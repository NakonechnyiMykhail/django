\documentclass[a4paper, 11pt]{extarticle}
%%% Работа с русским языком
\usepackage{cmap}					% поиск в PDF
\usepackage{mathtext} 				% русские буквы в формулах
\usepackage[T2A]{fontenc}			% кодировка
\usepackage[utf8]{inputenc}			% кодировка исходного текста
\usepackage[english,russian]{babel}	% локализация и переносы
%\usepackage[tmargin=4cm,bmargin=4cm,lmargin=2cm,rmargin=2cm,asymmetric,headheight=50pt,bindingoffset=1cm]{geometry}
\usepackage[tmargin=1.5cm,bmargin=1.5cm,lmargin=2cm,rmargin=1.5cm,headheight=50pt]{geometry}
\usepackage{comment} % enables the use of multi-line comments (\ifx \fi) 
\usepackage{lipsum} %This package just generates Lorem Ipsum filler text. 
\usepackage{fullpage} % changes the margin
%\usepackage[a4paper, total={7in, 10in}]{geometry}
\usepackage[fleqn]{amsmath}
\usepackage{amssymb,amsthm}  % assumes amsmath package installed
\newtheorem{theorem}{Theorem}
\newtheorem{corollary}{Corollary}
\usepackage{graphicx}
\usepackage{tikz}
\usetikzlibrary{arrows}
\usepackage{verbatim}
%\usepackage[numbered]{mcode}
\usepackage{float}
\usepackage{tikz}
    \usetikzlibrary{shapes,arrows}
    \usetikzlibrary{arrows,calc,positioning}

    \tikzset{
        block/.style = {draw, rectangle,
            minimum height=1cm,
            minimum width=1.5cm},
        input/.style = {coordinate,node distance=1cm},
        output/.style = {coordinate,node distance=4cm},
        arrow/.style={draw, -latex,node distance=2cm},
        pinstyle/.style = {pin edge={latex-, black,node distance=2cm}},
        sum/.style = {draw, circle, node distance=1cm},
    }
\usepackage{xcolor}
\usepackage{mdframed}
\usepackage[shortlabels]{enumitem}
%\usepackage{indentfirst}
\usepackage{hyperref}

\hypersetup{
    colorlinks=true,
    linkcolor=blue,
    filecolor=magenta,      
    urlcolor=cyan,
}
\renewcommand{\thesubsection}{\thesection.\alph{subsection}}

\newenvironment{problem}[2][Problem]
    { \begin{mdframed}[backgroundcolor=gray!20] \textbf{#1 #2} \\}
    {  \end{mdframed}}

% Define solution environment
\newenvironment{solution}
    {\textit{Solution:}}
    {}

\renewcommand{\qed}{\quad\qedsymbol}


\usepackage{fancyhdr}
\usepackage{emptypage}
\fancyhead{}
\fancyfoot{}
\fancyhead[LE,RO]{\includegraphics [width=.3\textwidth] {logo.png}}
\fancyfoot[LE,RO]{\thepage}
\pagestyle{fancy}
\graphicspath{{../../tex_templates/figures/}}

% code listing settings
\usepackage{listings}
\lstset{
    language=Python,
    basicstyle=\ttfamily\small,
    aboveskip={1.0\baselineskip},
    belowskip={1.0\baselineskip},
    columns=fixed,
    extendedchars=true,
    breaklines=true,
    tabsize=4,
    prebreak=\raisebox{0ex}[0ex][0ex]{\ensuremath{\hookleftarrow}},
    frame=lines,
    showtabs=false,
    showspaces=false,
    showstringspaces=false,
    keywordstyle=\color[rgb]{0.627,0.126,0.941},
    commentstyle=\color[rgb]{0.133,0.545,0.133},
    stringstyle=\color[rgb]{01,0,0},
    numbers=left,
    numberstyle=\small,
    stepnumber=1,
    numbersep=10pt,
    captionpos=t,
    escapeinside={\%*}{*)}
}



%%%%%%%%%%%%%%%%%%%%%%%%%%%%%%%%%%%%%%%%%%%%%%%%%%%%%%%%%%%%%%%%%%%%%%%%%%%%%%%%%%%%%%%%%%%%%%%%%%%%%%%%%%%%%%%%%%%%%%%%%%%%%%%%%%%%%%%%
\begin{document}
%Header-Make sure you update this information!!!!
%\noindent
%%%%%%%%%%%%%%%%%%%%%%%%%%%%%%%%%%%%%%%%%%%%%%%%%%%%%%%%%%%%

\noindent \LARGE{\textbf{Course: WebDev -- Django Start.}} \hfill  \\ 
%\vspace{0.25em}
\textbf{Homework - 3. Solve}  \hfill  \\
%Email: veralevel@gethu.edu \hfill ID: 123456789 \\

\noindent Theme: HTML Structure and CSS Styles. \hfill  \\
Level: beginner\\
Instructor: Mikhail Nakonechnyi \\
%Due Date: $28^{th}$ February, 2020 \\
\noindent\rule{6.257in}{2.8pt}
%%%%%%%%%%%%%%%%%%%%%%%%%%%%%%%%%%%%%%%%%%%%%%%%%%%%%%%%%%%%

\begin{problem}{1}
Основываясь на примере файла из классного занятия \\ \textbf{index.html}, сделать следующую структуру страницы. \\
\textit{\textbf{Примечание: }}весь текст можно заполнить <<рыбой>>. 
\begin{itemize}
\item <body>
\item <h1>Название блога</h1>
\begin{itemize}
\item <div>
\begin{itemize}
\item <h2>Название статьи 1</h2>
\item <h4>Короткое описание о чем статья</h4>
\item <p>Текст статьи</p>
\end{itemize}
\item </div>
\end{itemize}
\item </body>
\end{itemize}
На странице должно располагаться 5 статей, каждая статья внутри тэга <div>.
Количество символов в тексте должно быть в промежутке от 1000 до 1500.
\end{problem}
\begin{solution} 
%\vspace{-1.5em}
Решение смотреть в пункте [2].

%\vspace{-2.5em}
\end{solution} 
\noindent\rule{6.257in}{2.8pt}


%%%%%%%%%%%%%%%%%%%%%%%%%%%%%%%%%%%%%%%%%%%%%%%%%%%%%%%%%%%%

\begin{problem}{2}
Задать \textbf{цвет} фона глобально для всей страницы. Сделать стиль для \textit{названия блога}, выбрав \textbf{цвет} для фона и цвет для текста. Каждое \textit{название статьи} должно иметь одинаковый цвет фона и цвет текста. Тоже самое для описания и текста, между собой они одинаковые, но цвета должны быть уникальны.\\
\textit{\textbf{Примечание: }}то есть каждый тэг должен иметь уникальный цвет для фона и текста. Фон можно не каждому тэгу задавать, но минимум 2. Для глубокого понимания и выполнения этой задачи, см. пункт [4]. %\ref{problem4}.
\end{problem}
\begin{solution} 
\vspace{-1.5em}
\lstinputlisting[language=HTML,caption=index.html.]{index.html}
\vspace{-0.5em}
\end{solution} 
\noindent\rule{6.257in}{2.8pt}
%%%%%%%%%%%%%%%%%%%%%%%%%%%%%%%%%%%%%%%%%%%%%%%%%%%%%%%%%%%%

\begin{problem}{3}
Загрузить все файлы на \textbf{GitHub}. Отправить мне ссылку на ваш репозиторий в telegram.
\end{problem}
\begin{solution} 
\begin{lstlisting}[language=Bash]
cd ~/Documents/DjangoStart/
git status
git pull
git add lesson3/
git commit -m "homework 3"
git push -u origin master
\end{lstlisting}
\end{solution} 
\noindent\rule{6.257in}{2.8pt}
%%%%%%%%%%%%%%%%%%%%%%%%%%%%%%%%%%%%%%%%%%%%%%%%%%%%%%%%%%%%

\begin{problem}{4}\label{problem4}
Прочитать \href{https://html5book.ru/html-html5}{html5book}, пункты 1.1-1.6 (включительно). Из того, что проходили - \textbf{понимать все!!!}, с остальным главное ознакомиться.\\
Для практики и примеров можно использовать \href{https://www.w3schools.com}{w3school}. 
\end{problem}
\noindent\rule{6.257in}{2.8pt}



\end{document}
 